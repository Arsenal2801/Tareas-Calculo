\documentclass{article}
%=======================================================
%                      Dependencies
%=======================================================
\usepackage{graphicx}

\begin{document}
%=======================================================
%                      Front Page
%=======================================================
  \begin{titlepage}
    {\includegraphics[width=0.3\textwidth]{multimedia/ESCOM-Logo.png}}
    \hfill
    {\includegraphics[width=0.15\textwidth]{multimedia/Logo_Instituto_Politécnico_Nacional.png}\par}
    \vspace{1cm}
    \centering
    {\bfseries\LARGE  INSTITUTO POLIT\'ECNICO NACIONAL\par}
    \vspace{1cm}
    {\scshape\LARGE Escuela Superior de C\'omputo\par}
    \vspace{1cm}
    {\scshape\Huge T\'itulo del proyecto\par}
    \vspace{2cm}
    {\itshape\Large Zarate Cardenas Alejandro\par}
    \vfill
    {\Large Autores:\par}
    {\Large Campos Zeron Salvador\par}
    {\Large Díaz González Lizeth\par}
    {\Large Girón Flores Carlos Alberto\par}
    {\Large Hernández García Jaime Gabriel\par}
    {\Large Izoteco Zacarias Pedro Uriel\par}
    {\Large S\'anchez Ortega Gabriel\par}
    \vfill
    {\Large 10 de Marzo 2023 \par}
  \end{titlepage}

  %=======================================================
  %                         Content
  %=======================================================
  \renewcommand*\contentsname{Índice}
  \tableofcontents
  \newpage
  \section{Introducción}
    El capítulo 5 del libro "Cálculo de Trascendentes Tempranas" de Dennis G. Zill se enfoca en el tema de las integrales definidas. Las integrales definidas son una herramienta fundamental en el cálculo integral y se utilizan para calcular el área bajo una curva, el volumen de sólidos de revolución y muchas otras aplicaciones en matemáticas, física y otras áreas de la ciencia.
  \subsection{Lo que revisamos}
    A lo largo de este documento se podra ver la resolucion de los problemas propuestos en el libro anteriormente mencionado aclarando que dichos problemas fueron divididos por equipo tocandole al "Equipo Capibara" los problemas impares(5,17,29,41,53,65) de las paginas: 293-295 y 303-305
  
\end{document}