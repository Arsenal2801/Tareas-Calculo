\documentclass{article}
%=======================================================
%                      Dependencies
%=======================================================
\usepackage{graphicx}
\usepackage{multicol}
\usepackage{amsmath}
\usepackage{pgfplots}

\begin{document}
%=======================================================
%                      Front Page
%=======================================================
  \begin{titlepage}
    {\includegraphics[width=0.3\textwidth]{multimedia/ESCOM-Logo.png}}
    \hfill
    {\includegraphics[width=0.15\textwidth]{multimedia/Logo_Instituto_Politécnico_Nacional.png}\par}
    \vspace{1cm}
    \centering
    {\bfseries\LARGE  INSTITUTO POLIT\'ECNICO NACIONAL\par}
    \vspace{1cm}
    {\scshape\LARGE Escuela Superior de C\'omputo\par}
    \vspace{1cm}
    {\scshape\Huge Unidad 1: La integral definida y sus aplicaciones \par}
    \vspace{2cm}
    {\itshape\Large Zarate Cardenas Alejandro\par}
    \vfill
    {\Large Autores:\par}
    {\Large Campos Zeron Salvador\par}
    {\Large Díaz González Lizeth\par}
    {\Large Girón Flores Carlos Alberto\par}
    {\Large Hernández García Jaime Gabriel\par}
    {\Large Izoteco Zacarias Pedro Uriel\par}
    {\Large S\'anchez Ortega Gabriel\par}
    \vfill
    {\Large 10 de Marzo 2023 \par}
  \end{titlepage}

  %=====================================================
  %                         Content
  %=====================================================
  \renewcommand*\contentsname{Índice}
  \tableofcontents
  \newpage
  \section{Introducción}

  La integral definida es uno de los conceptos fundamentales en el cálculo, una rama de las matemáticas que estudia el cambio y la variación. La integral definida es una operación matemática que permite calcular el valor de una función en un intervalo específico. A diferencia de la integral indefinida, que devuelve una familia de funciones, la integral definida da como resultado un número único.

  La integral definida tiene múltiples aplicaciones en diversas áreas de la ciencia, la ingeniería y la economía. En física, la integral definida se utiliza para calcular la energía cinética y potencial, la velocidad y la aceleración de un objeto en movimiento. En ingeniería, se utiliza para diseñar puentes, carreteras y edificios, y para optimizar procesos industriales. En economía, la integral definida se utiliza para calcular el valor presente de una inversión o el flujo de caja de una empresa. En biología, se utiliza para modelar el crecimiento de poblaciones y la distribución de especies en un ecosistema.
  
  Para calcular una integral definida, se divide el intervalo en pequeñas secciones llamadas subintervalos y se aproxima el valor de la función en cada uno de ellos. Luego, se suman los resultados de cada subintervalo para obtener una aproximación del valor de la integral. A medida que se dividen los subintervalos en secciones más pequeñas, la aproximación se vuelve más precisa.
  
  En resumen, la integral definida es una herramienta matemática poderosa con innumerables aplicaciones prácticas en diversas áreas. Es una herramienta fundamental en el cálculo y se utiliza para calcular el área bajo una curva, la longitud de una curva y el volumen de un sólido de revolución, entre otras cosas. Su importancia en la ciencia y la ingeniería es crucial para el avance y desarrollo de la tecnología moderna.
  \subsection{Lo que revisamos}
    A lo largo de este documento se podra ver la resolucion de los problemas propuestos en el libro anteriormente mencionado aclarando que dichos problemas fueron divididos por equipo tocandole al "Equipo Capibara" los problemas impares(5,17,29,41,53) de las paginas: 293-295, 303-305, 313-314, 331-332, 430-432 
  \section{Problemario 1: 293-295}
  
      \subsection{Problema 5}
        Desarrolle la suma indicada:
       \begin{equation}\label{eq_5.1}
        \sum_{k=1}^{10}\frac{(-1)^k}{2k+5}
       \end{equation}
       Donde si hacemos a sumatoria dada por la ecuacion \ref{eq_5.1} tenemos:
       \begin{equation}\label{eq_5.2}
        \begin{split}
          \frac{(-1)^1}{2(1)+5}+\frac{(-1)^2}{2(2)+5}+\frac{(-1)^3}{2(3)+5}+\frac{(-1)^4}{2(4)+5}+\frac{(-1)^5}{2(5)+5}+\\\frac{(-1)^6}{2(6)+5}+\frac{(-1)^7}{2(7)+5}+\frac{(-1)^8}{2(8)+5}+\frac{(-1)^9}{2(9)+5}+\frac{(-1)^{10}}{2(10)+5}
        \end{split}
       \end{equation}
       Realizando las respectivas operaciones en \ref{eq_5.2}
       \begin{equation}\label{eq_5.3}
          -\frac{1}{7}+\frac{1}{9}-\frac{1}{11}+\frac{1}{13}-\frac{1}{15}+\frac{1}{17}-\frac{1}{19}+\frac{1}{21}-\frac{1}{23}+\frac{1}{25} = -0.062065975\dots
       \end{equation}
       Teniendo asi que el resultado esta dado en la ecuacion \ref{eq_5.4}, es decir, la que esta a continuacion: 
       \begin{equation}\label{eq_5.4}
        \sum_{k=1}^{10}\frac{(-1)^k}{2k+5} = -0.062065975\dots
       \end{equation}
      \subsection{Problema 17}
       Escriba en notación sigma la suma dada
       \setcounter{equation}{0}
       \begin{equation}\label{eq_17.1}
        6+6+6+6+6+6+6+6
       \end{equation}
       Contando el numero de veces que aparce el 6 en la suma \ref{eq_17.1} tenemos que k(contador) toma valores de 1 a 8 dejandonos con la siguiente expresion:
       \begin{equation}\label{eq_17.2}
        \sum_{k=1}^{8}6
       \end{equation}
       \setcounter{equation}{0}
      \subsection{Problema 29}
        Utilizando lo siguiente:
        \begin{equation} \label{prop_area}
          A = \lim_{n\to\infty} \sum_{k=1}^{n} f\left(a+k\frac{b-a}{n}\right)* \frac{b-a}{n}
        \end{equation}
        Y los siguientes teoremas:
        \begin{align*}
          \sum_{k=1}^{n} c &= nc & \sum_{k=1}^{n} k &= \frac{n(n+1)}{2} \\ \sum_{k=1}^{n} k^2 &= \frac{n(n+1)(2n+1)}{6} & \sum_{k=1}^{n} k^3 &= \frac{n^2(n+1)^2}{4}
        \end{align*}
        Con la propiedad dada en la ecuacion \ref{prop_area} y los teoremas dados anteriormente calcule el area bajo la funcion siguiente:
        \begin{equation}
          f(x) = x, [0,6]
        \end{equation}
        Con dicha informacion podemos decir que:
        \begin{align*}
          a &= 0 & b &= 6
        \end{align*}
        Entonces:
        \begin{equation}
          A = \lim_{n\to\infty} \sum_{k=1}^{n} f\left(0+k\frac{6-0}{n}\right)* \frac{6-0}{n}
        \end{equation}
        Desarrollando:
        \begin{align*}
          A = \lim_{n\to\infty} \sum_{k=1}^{n} f\left(k\frac{6}{n}\right)* \frac{6}{n} =\dots\\\dots = \lim_{n\to\infty} \sum_{k=1}^{n} \left(\frac{36k}{n^2}\right)=\lim_{n\to\infty}\left[\frac{36}{n^2} \sum_{k=1}^{n} k\right]=\dots \\ \dots = \lim_{n\to\infty}\left[\frac{36}{n^2} *\frac{n(n+1)}{2}\right]=\lim_{n\to\infty} \left[18*\frac{n(n+1)}{n^2}\right]=\dots \\ \dots = 18\lim_{n\to\infty} \left[\frac{n+1}{n}\right]= 18*1=18
        \end{align*}
        Por lo tanto $A = 18$
      \subsection{Problema 41}
        Utilizando los teoremas y la definicion del problema anteriror resuelva lo siguiente:
        \begin{equation*}
          f(x) = \left\{
            \begin{array}{lcc}
              2, & si & 0\leq x < 1 \\
              x+1, & si & 1 \leq x \leq 4
            \end{array}
          \right.
        \end{equation*}
        Entonces podemos decir que el area esta dada por:
        \begin{equation}
          A = \int_{0}^{1} f(x)dx + \int_{1}^{5} f(x)dx
        \end{equation}
        Gracias a esta suma podemos trabajar por separado los dos trozos de la funcion. Asi facilitando s manipulacion.
        Para el segmento de $\int_{0}^{1}f(x)dx$ la funcion esta evaluada en 2. Por lo cual el area esta dada por un valor constante y para este caso debemos agregar una propiedad:
        \begin{equation}
          \int_{a}^{b}kdx = k\int_{a}^{b}dx = k(b-a)
        \end{equation}
        Sustituyendo:
        \begin{equation}
          A_1=\int_{0}^{1}2dx = 2\int_{0}^{1}dx = 2(1-0)=2
        \end{equation}
        Trabajando con la segunda parte tenemos:
        \begin{equation}
          A_2 = \int_{1}^{4} f(x)dx = \int_{1}^{4}xdx + \int_{1}^{4}1dx
        \end{equation}
        Trabajando por partes:
        \begin{align*}
          i) \int_{1}^{4}xdx&=\dots & ii) 1\int_{1}^{4}dx&= \dots \\ \dots=\lim_{n\to\infty} \sum_{k=1}^{n}\left[1+\frac{3k}{n}\right]\cdot\frac{3}{n}&=\dots & \dots = 1(4-1)&=3 \\
          \dots=\lim_{n\to\infty} \sum_{k=1}^{n}\left[\frac{3}{n}+\frac{9k}{n^2}\right]\cdot\frac{3}{n}&=\dots \\ \dots = \lim_{n\to\infty}\left[3+\frac{9}{n^2}\cdot\frac{n(n+1)}{2}\right] = \dots \\ \dots=\lim_{n\to\infty}\left[3+\frac{9}{2}\cdot\frac{n+1}{n}\right]= 3+\frac{9}{2} = \frac{15}{2}
        \end{align*}
        Ya que se tienen las integrales por partes tenemos lo siguiente:
        \begin{equation}
          A_2 = \int_{1}^{4}xdx = \frac{15}{2} + 3 = \frac{21}{2}
        \end{equation}
        Para terminar realizamos la suma entre \(A_1\) y \(A_2\) para obtener \(A_T\):
        \begin{equation}
          A_T = \int_{0}^{4} f(x)dx = A_1 + A_2 = \frac{21}{2} + 2 = \frac{25}{2}
        \end{equation}
      \subsection{Problema 53}
      Despeje \(\bar{x}\) de :
      \begin{equation}
        \sum_{k=1}^{n}(x_k-\bar{x})^2 = 0
      \end{equation}
      Desarrollando: 
      \begin{align*}
        \sum_{k=1}^{n}x^2_k - 2x_k\bar{x} + \bar{x}^2=0 \\ \sum_{k=1}^{n}x^2_k -\sum_{k=1}^{n} 2x_k\bar{x} + \sum_{k=1}^{n} \bar{x}^2 = 0 \\ \sum_{k=1}^{n} x^2_k -2\bar{x}\sum_{k=1}^{n}x_k + \bar{x}^2=0 \\ \bar{x}^2-2\bar{x}\sum_{k=1}^{n}x_k = -\sum_{k=1}^{n} x^2_k \\ \bar{x}^2-2\bar{x}\sum_{k=1}^{n}x_k + \left(\sum_{k=1}^{n} x_k\right)^2 = \left(\sum_{k=1}^{n} x_k\right)^2 -\sum_{k=1}^{n} x^2_k \\ \left(\bar{x} + \sum_{k=1}^{n} x_k \right)^2 = \left(\sum_{k=1}^{n} x_k\right)^2 -\sum_{k=1}^{n} x^2_k \\ \bar{x} + \sum_{k=1}^{n} x_k = \sqrt{\left(\sum_{k=1}^{n} x_k\right)^2 -\sum_{k=1}^{n} x^2_k} \\ \bar{x} = \sqrt{\left(\sum_{k=1}^{n} x_k\right)^2 -\sum_{k=1}^{n} x^2_k} - \sum_{k=1}^{n} x_k
      \end{align*}
      \subsection{Conclusión}
      En conclusión, las sumas de Riemann son una técnica matemática utilizada para aproximar el valor de una integral definida. Esta técnica se basa en dividir el intervalo de integración en subintervalos y aproximar el valor de la función en cada uno de ellos. A medida que se divide el intervalo en secciones más pequeñas, la aproximación se vuelve más precisa. Las sumas de Riemann son una herramienta esencial en el cálculo y tienen aplicaciones en diversas áreas, como la física, la ingeniería y la economía. Además, son la base para el desarrollo del cálculo integral, una rama fundamental de las matemáticas.
      \section{Problemario 2: 303-305}
      \subsection{Problema 5}
      Sea una funcion \(f(x)\) definida por: \(f(x) = sin(x)\) calcule el area dada por el intervalo cerrado: \([0,2\pi]\) considerando los siguientes subintervalos: $$x_0 = 0; x_1 = \pi; x_2 = \frac{3\pi}{2}; x_3 = 2\pi$$ y: $$x_1^*=\frac{\pi}{2}; x_2^*=\frac{7\pi}{6}; x_3^* = \frac{7\pi}{4}$$
      Como ya sabemos el area se calcula con: $$\sum_{k=1}^{n}f(x^*_k) \bigtriangleup  x_k$$
      Por lo que desarrollando tenemos:
      \begin{align*}
        \sum_{k=1}^{3}f(x^*_k) \bigtriangleup  x_k && = && f(x_1^*)\bigtriangleup x_1 + f(x_2^*)\bigtriangleup x_2 + f(x_3^*)\bigtriangleup x_3 \\
        f(x_1^*) = sin\left(\frac{\pi}{2}\right) = 1 && ; && \bigtriangleup x_1 = \pi - 0 = \pi  \\
        f(x_2^*) = sin\left(\frac{7\pi}{6}\right) = -\frac{1}{2} && ; && \bigtriangleup x_2 = \frac{3\pi}{2} - \pi = \frac{\pi}{2} \\ 
        f(x_3^*) = sin\left(\frac{7\pi}{4}\right) = -\frac{\sqrt{2}}{2} && ; && \bigtriangleup x_3 = 2\pi - \frac{3\pi}{2}  = \frac{\pi}{2}
      \end{align*}
      Sustituyendo:
      \begin{align*}
        \sum_{k=1}^{3}f(x^*_k) \bigtriangleup  x_k = f(x_1^*)\bigtriangleup x_1 + f(x_2^*)\bigtriangleup x_2 + f(x_3^*)\bigtriangleup x_3 = \dots \\ \dots = \pi - \frac{1}{2} \cdot \frac{\pi}{2} - \frac{\sqrt{2}}{2} \cdot \frac{\pi}{2} \approx 1.245473
      \end{align*}
      \subsection{Problema 17}
      Integre por definicion lo siguiente:
      $$\int_{0}^{1}(x^3 -1)dx$$
      Utilizando las formulas dadas en el problema 29 de la seccion anterior y algunas propiedades de las integrales tenemos:
      \begin{align*}
        \int_{0}^{1} x^3dx + \int_{0}^{1} 1dx
      \end{align*}
      Trabajando por separado:
      \begin{align*}
        \int_{0}^{1} x^3dx =\lim_{n\to\infty} \sum_{k=1}^{n}f\left(\frac{k}{n}\right)\cdot\frac{1}{n} = \dots && -\int_{0}^{1}1dx = -1(1-0) = -1 \\
        \dots = \lim_{n\to\infty} \sum_{k=1}^{n} \left(\frac{k^3}{n^4}\right) = \lim_{n\to\infty} \left[\frac{1}{n^4}\sum_{k=1}^{n}k^3\right]=\dots \\ \dots = \lim_{n\to\infty}\left[\frac{1}{4}\cdot\frac{n^2(n+1)^2}{n^4}\right] = \lim_{n\to\infty}\left[\frac{1}{4}\cdot\frac{(n+1)^2}{n^2}\right]= \dots \\ \dots = \frac{1}{4} \lim_{n\to\infty}\left[\frac{n^2+2n+1}{n^2}\right] = \frac{1}{4} \cdot 1 = \frac{1}{4}
      \end{align*}
      Realizando la suma de las 2 partes:
      \begin{align*}
        \int_{0}^{1}(x^3 -1)dx = \int_{0}^{1} x^3dx + \int_{0}^{1} 1dx  = \frac{1}{4} -1 = -\frac{3}{4}
      \end{align*}
      \subsection{Problema 29}
      Integre:
      $$\int_{3}^{-1}t^2dt$$
      Desarrollando con las propiedades anteriores:
      \begin{align*}
        \int_{3}^{-1} t^2 dt = \lim_{n\to\infty}\sum_{k=1}^{n}f\left(3-\frac{4k}{n}\right)\cdot\left(-\frac{4}{n}\right) =\dots \\\dots = \lim_{n\to\infty}\sum_{k=1}^{n}\left(9-\frac{24k}{n}+\frac{16k^2}{n^2}\right)\cdot\left(-\frac{4}{n}\right) = \dots \\ \dots = \lim_{n\to\infty}\sum_{k=1}^{n}\left[-\frac{36}{n}+\frac{96k}{n^2}-\frac{64k^2}{n^3}\right] = \dots \\ \dots = \lim_{n\to\infty}\left[-\frac{36}{n}+\frac{96}{n^2}\cdot\frac{n(n+1)}{2}-\frac{64}{n^3}\cdot\frac{n(n+1)(2n+1)}{6}\right]=\dots \\ \dots = \lim_{n\to\infty}\left[-36+48\cdot\frac{n+1}{n}-\frac{32}{3}\cdot\frac{2n^2+3n+1}{n^2}\right]=\dots \\ \dots = -36+8-\frac{64}{3} = - \frac{28}{3}
      \end{align*}
      \subsection{Problema 41}
      Calcule:
      $$
      \int_{-1}^{2} [2f(x)+g(x)]dx
      $$
      Si
      \begin{align*}
        \int_{-1}^{2}f(x)dx = 3.4 && y && 3\int_{-1}^{2}g(x)dx=12.6
      \end{align*}
      Haciendo las operaciones respectivas:
      \begin{align*}
        2\int_{-1}^{2}f(x)dx + \int_{-1}^{2}g(x)dx = 2\cdot3.4 + \frac{12.6}{3} = 6.8 + 4.2 = 11
      \end{align*}
      \subsection{Problema 53}
      La integral dada representa la siguiente area con signo entre una grafica y el eje x sobre un intervalo. Trace esta region. \newline
      \begin{center}
        \begin{tikzpicture}
          \begin{axis}[
              xlabel={$x$},
              ylabel={$y$},
              xmin=-5, xmax=5,
              ymin=-10, ymax=10,
              axis lines=middle,
              samples=100,
              domain=-5:5,
          ]
          
          % graficar la recta -2x+6
          \addplot[color=blue,thick]{-2*x+6};
          
          % remarcar el área bajo la curva en el intervalo [0,5]
          \addplot[fill=green,opacity=0.5,domain=0:5] {-2*x+6} \closedcycle;
          
          \node[above right, font=\fontsize{12}{15}\selectfont\bfseries] at (700,150) {A = 5u²};
  
          \end{axis}
        \end{tikzpicture}
      \end{center}
      \subsection{Problema 65}
      Utilizando propiedades de comparacion demuestre que:
      $$\int_{-1}^{0}e^xdx\leq \int_{-1}^{0}e^{-x}dx$$
      Si \(g(x)=e^x\), entonces \(\int_{-1}^{1}g(x)dx=1-\frac{1}{e}\) por lo tanto: \(\int_{-1}^{0}e^{-x}\geq 1-\frac{1}{e}\).
      \newline
      De forma semejante tomando \(f(x) = e^{-x}\), entonces \(\int_{-1}^{0}f(x)dx = e-1\), por lo tanto \(\int_{-1}^{0}e^xdx\leq e-1\). \newline
      Con esta informacion podemos decir que:
      $$
      e-1 \geq 1- \frac{1}{e}
      $$
      Y esta desigualdad se cumple ya que sustituyendo con sus valores aproximados tenemos:
      $$1.7182\geq 0.6321$$
      Por lo tanto esto es verdadero.
\end{document}